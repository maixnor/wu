% Options for packages loaded elsewhere
\PassOptionsToPackage{unicode}{hyperref}
\PassOptionsToPackage{hyphens}{url}
%
\documentclass[
]{article}
\usepackage{amsmath,amssymb}
\usepackage{iftex}
\ifPDFTeX
  \usepackage[T1]{fontenc}
  \usepackage[utf8]{inputenc}
  \usepackage{textcomp} % provide euro and other symbols
\else % if luatex or xetex
  \usepackage{unicode-math} % this also loads fontspec
  \defaultfontfeatures{Scale=MatchLowercase}
  \defaultfontfeatures[\rmfamily]{Ligatures=TeX,Scale=1}
\fi
\usepackage{lmodern}
\ifPDFTeX\else
  % xetex/luatex font selection
\fi
% Use upquote if available, for straight quotes in verbatim environments
\IfFileExists{upquote.sty}{\usepackage{upquote}}{}
\IfFileExists{microtype.sty}{% use microtype if available
  \usepackage[]{microtype}
  \UseMicrotypeSet[protrusion]{basicmath} % disable protrusion for tt fonts
}{}
\makeatletter
\@ifundefined{KOMAClassName}{% if non-KOMA class
  \IfFileExists{parskip.sty}{%
    \usepackage{parskip}
  }{% else
    \setlength{\parindent}{0pt}
    \setlength{\parskip}{6pt plus 2pt minus 1pt}}
}{% if KOMA class
  \KOMAoptions{parskip=half}}
\makeatother
\usepackage{xcolor}
\usepackage[margin=1in]{geometry}
\usepackage{color}
\usepackage{fancyvrb}
\newcommand{\VerbBar}{|}
\newcommand{\VERB}{\Verb[commandchars=\\\{\}]}
\DefineVerbatimEnvironment{Highlighting}{Verbatim}{commandchars=\\\{\}}
% Add ',fontsize=\small' for more characters per line
\usepackage{framed}
\definecolor{shadecolor}{RGB}{248,248,248}
\newenvironment{Shaded}{\begin{snugshade}}{\end{snugshade}}
\newcommand{\AlertTok}[1]{\textcolor[rgb]{0.94,0.16,0.16}{#1}}
\newcommand{\AnnotationTok}[1]{\textcolor[rgb]{0.56,0.35,0.01}{\textbf{\textit{#1}}}}
\newcommand{\AttributeTok}[1]{\textcolor[rgb]{0.13,0.29,0.53}{#1}}
\newcommand{\BaseNTok}[1]{\textcolor[rgb]{0.00,0.00,0.81}{#1}}
\newcommand{\BuiltInTok}[1]{#1}
\newcommand{\CharTok}[1]{\textcolor[rgb]{0.31,0.60,0.02}{#1}}
\newcommand{\CommentTok}[1]{\textcolor[rgb]{0.56,0.35,0.01}{\textit{#1}}}
\newcommand{\CommentVarTok}[1]{\textcolor[rgb]{0.56,0.35,0.01}{\textbf{\textit{#1}}}}
\newcommand{\ConstantTok}[1]{\textcolor[rgb]{0.56,0.35,0.01}{#1}}
\newcommand{\ControlFlowTok}[1]{\textcolor[rgb]{0.13,0.29,0.53}{\textbf{#1}}}
\newcommand{\DataTypeTok}[1]{\textcolor[rgb]{0.13,0.29,0.53}{#1}}
\newcommand{\DecValTok}[1]{\textcolor[rgb]{0.00,0.00,0.81}{#1}}
\newcommand{\DocumentationTok}[1]{\textcolor[rgb]{0.56,0.35,0.01}{\textbf{\textit{#1}}}}
\newcommand{\ErrorTok}[1]{\textcolor[rgb]{0.64,0.00,0.00}{\textbf{#1}}}
\newcommand{\ExtensionTok}[1]{#1}
\newcommand{\FloatTok}[1]{\textcolor[rgb]{0.00,0.00,0.81}{#1}}
\newcommand{\FunctionTok}[1]{\textcolor[rgb]{0.13,0.29,0.53}{\textbf{#1}}}
\newcommand{\ImportTok}[1]{#1}
\newcommand{\InformationTok}[1]{\textcolor[rgb]{0.56,0.35,0.01}{\textbf{\textit{#1}}}}
\newcommand{\KeywordTok}[1]{\textcolor[rgb]{0.13,0.29,0.53}{\textbf{#1}}}
\newcommand{\NormalTok}[1]{#1}
\newcommand{\OperatorTok}[1]{\textcolor[rgb]{0.81,0.36,0.00}{\textbf{#1}}}
\newcommand{\OtherTok}[1]{\textcolor[rgb]{0.56,0.35,0.01}{#1}}
\newcommand{\PreprocessorTok}[1]{\textcolor[rgb]{0.56,0.35,0.01}{\textit{#1}}}
\newcommand{\RegionMarkerTok}[1]{#1}
\newcommand{\SpecialCharTok}[1]{\textcolor[rgb]{0.81,0.36,0.00}{\textbf{#1}}}
\newcommand{\SpecialStringTok}[1]{\textcolor[rgb]{0.31,0.60,0.02}{#1}}
\newcommand{\StringTok}[1]{\textcolor[rgb]{0.31,0.60,0.02}{#1}}
\newcommand{\VariableTok}[1]{\textcolor[rgb]{0.00,0.00,0.00}{#1}}
\newcommand{\VerbatimStringTok}[1]{\textcolor[rgb]{0.31,0.60,0.02}{#1}}
\newcommand{\WarningTok}[1]{\textcolor[rgb]{0.56,0.35,0.01}{\textbf{\textit{#1}}}}
\usepackage{graphicx}
\makeatletter
\def\maxwidth{\ifdim\Gin@nat@width>\linewidth\linewidth\else\Gin@nat@width\fi}
\def\maxheight{\ifdim\Gin@nat@height>\textheight\textheight\else\Gin@nat@height\fi}
\makeatother
% Scale images if necessary, so that they will not overflow the page
% margins by default, and it is still possible to overwrite the defaults
% using explicit options in \includegraphics[width, height, ...]{}
\setkeys{Gin}{width=\maxwidth,height=\maxheight,keepaspectratio}
% Set default figure placement to htbp
\makeatletter
\def\fps@figure{htbp}
\makeatother
\setlength{\emergencystretch}{3em} % prevent overfull lines
\providecommand{\tightlist}{%
  \setlength{\itemsep}{0pt}\setlength{\parskip}{0pt}}
\setcounter{secnumdepth}{-\maxdimen} % remove section numbering
\ifLuaTeX
  \usepackage{selnolig}  % disable illegal ligatures
\fi
\IfFileExists{bookmark.sty}{\usepackage{bookmark}}{\usepackage{hyperref}}
\IfFileExists{xurl.sty}{\usepackage{xurl}}{} % add URL line breaks if available
\urlstyle{same}
\hypersetup{
  hidelinks,
  pdfcreator={LaTeX via pandoc}}

\author{}
\date{\vspace{-2.5em}}

\begin{document}

\thispagestyle{empty}

\begin{center}
  \vspace*{2cm}
  \Huge\textbf{Case Study 1 - Probability} \\[0.5cm]
  \Large\textbf{By example of Sports Team Selection Procedures} \\[1cm]
  \large
  \begin{tabular}{c}
    Benjamin Maixner \\
    Chirill Glitos \\
    Franziska Kirschner \\
    Said Rizvanov \\
  \end{tabular}
  \vfill
  \normalsize
  \textbf{Date}: Juni 21, 2024
\end{center}

\newpage

\section{1 Assembling the Team}\label{assembling-the-team}

The goal is to assemble a team of 5 athletes. The athletes are randomly
picked from the available pool (6 from country A and 7 from country B).
But there are certain conditions to be met each time. How the conditions
will influence the success of our selection process is explored here.

\subsubsection{a) Exact Ratio}\label{a-exact-ratio}

When selecting any 3 from country A and selecting any 2 from country B
giving a split of 3 to 2:

\[
P(3A + 2B) = \frac{\binom{6}{3} \times \binom{7}{2}}{\binom{13}{5}} = 0.\overline{326340} \approx 32.63\%
\]

\subsubsection{b) Just More}\label{b-just-more}

When there should just be more athletes of country B than from A.
Necessitating athletes from B being more or equal to 3:

\[
P(B > A) = P(B \geq 3) = \frac{\binom{7}{3} \times \binom{6}{2} + \binom{7}{4} \times \binom{6}{1} + \binom{7}{5} \times \binom{6}{0}}{\binom{13}{5}} = 0.587412 \approx 58.74\%
\]

\subsubsection{c) At least 1}\label{c-at-least-1}

When there needs to be at least 1 athlete from country A we can equally
exclude the option with no athlete from A:

\[
P(A \geq 1) = 1 - P(A = 0) = 1- \frac{\binom{7}{5} \times \binom{6}{0}}{\binom{13}{5}} = 0.983682 \approx 98.37\%
\]

\subsubsection{d) Be Specific Please}\label{d-be-specific-please}

When we need to select 1 specific athlete from country A (Alex) and
country B
(Bettina)\footnote{First character of names matching with the country name is purely coincidental}:

\[
P(\text{Alex} \cap \text{Bettina} \in \text{Team}) =\frac{\binom{11}{3}}{\binom{13}{5}} = \frac{165}{1287} \approx 12.82\%
\]

\section{2 Basket or Foot?}\label{basket-or-foot}

The team is divided into basketball and football fans, where basketball
fans always vote for the same TV channel and football fans may change
their choice daily with a probability \(q\). We need to calculate the
probability that any randomly selected member will vote the same way as
the previous day.

Given:

\begin{itemize}
    \item Fraction \( p \) of the team are basketball fans who always vote the same.
    \item Fraction \( 1-p \) are football fans, and they change their choice with probability \( q \).
\end{itemize}

The probability that a basketball fan votes the same way again is 1
(since they never change), and the probability that a football fan votes
the same way is \(1-q\) (they vote differently with probability \(q\)).

Therefore, the overall probability that a member votes the same way as
they did the previous time is calculated by weighing these probabilities
by the proportion of each type of fan:

\[
P(\text{same}) = p + (1-p)(1-q)
\]

This formula provides the expected consistency in voting behavior across
the team, considering the behavior patterns of both subgroups of fans.

\section{3 Time to Perform}\label{time-to-perform}

Each member of the team is supposed to supervise another during
training. Who supervises who is supposed to be decided using a bowl and
some paper. Each person will pick a random name from the bowl. Since
self-supervision generally doesn't work well, and an athlete is not
supposed to pick (and therefore supervise) themself.

The worst case would be each person picking their own name, then the
whole operation was done for nothing!

\[
P(\text{Picking was useless}) = \frac{1}{5} \times \frac{1}{4} \times \frac{1}{3} \times \frac{1}{2} \times \frac{1}{1} = \frac{1}{120} = 0.008\overline{3333} = \approx 0.83\%
\]

\section{4 Extrapolation Time}\label{extrapolation-time}

Now we are tasked with creating a general model for this predicament.
For larger \(n\) the method of writing out the individual chances are
impractical\footnote{Actually, even 5 fractions to type before was testing my patience}.

\subsubsection{a) Hurting Pinky}\label{a-hurting-pinky}

Now if we take a closer, lazier look at the last expression we will find
a pattern which is strangely familiar. The factorial! This also saves
the pinky from typing all these curly braces.

\[
... = \frac{1}{5} \times \frac{1}{4} \times \frac{1}{3} \times \frac{1}{2} \times \frac{1}{1} = \frac{1}{5 * 4 * 3 * 2 * 1} = \frac{1}{n!}
\]

\subsubsection{b) Favorite Athletes}\label{b-favorite-athletes}

Since we recognized that all athletes getting their own names in the
selection procedure is quite rare we ask ourselves the question of a
subset of athletes getting their own names back. The first \(m\)
athletes are supposed to be assigned other athletes. The rest (\(n-m\))
of the team we don't talk about\ldots{}

\[
P(\text{first } m) = \frac{m! \times (n-m)!}{n!}
\]

\subsubsection{c) Team Inception}\label{c-team-inception}

Now we shall divide the team into 2 groups wherby the first \(m\)
members should be assigned people of the other part. The larger \(m\) is
the lower the probability of succeeding.

\[
P(\text{switch}) = \frac{\binom{n}{m} \times m!}{n!}
\]

\subsubsection{d) Dirty Hands}\label{d-dirty-hands}

Considering the probability \(p\) that each paper might get dirty, the
probability that the first \(m\) members all pick up clean pieces of
paper is just the probability raised to the power of \(m\)

\[
P(\text{clean}) = (1-p)^m
\]

\section{5 Codify the Drawing Process}\label{codify-the-drawing-process}

\subsubsection{Rinse and Repeat}\label{rinse-and-repeat}

But how to solve the bowl and paper problem in a codified manner? For
this we will leverage the inbuilt \texttt{sample()} function of R to
shuffle the names vector. Should any name be at the same index we
shuffle again. This might be repeated multiple times until all names are
at different indexes.

\begin{Shaded}
\begin{Highlighting}[]
\CommentTok{\# Function to draw names such that no one picks their own name}
\NormalTok{draw\_names }\OtherTok{\textless{}{-}} \ControlFlowTok{function}\NormalTok{(names\_vector) \{}
\NormalTok{  n }\OtherTok{\textless{}{-}} \FunctionTok{length}\NormalTok{(names\_vector)  }\CommentTok{\# Get the number of names}
\NormalTok{  shuffled\_names }\OtherTok{\textless{}{-}} \FunctionTok{sample}\NormalTok{(names\_vector)  }\CommentTok{\# Randomly shuffle the names}

  \CommentTok{\# Continue shuffling until no one has their own name}
  \ControlFlowTok{while}\NormalTok{(}\FunctionTok{any}\NormalTok{(shuffled\_names }\SpecialCharTok{==}\NormalTok{ names\_vector)) \{}
\NormalTok{    shuffled\_names }\OtherTok{\textless{}{-}} \FunctionTok{sample}\NormalTok{(names\_vector)}
\NormalTok{  \}}

  \FunctionTok{return}\NormalTok{(shuffled\_names)  }\CommentTok{\# Return the valid permutation}
\NormalTok{\}}

\CommentTok{\# Example usage:}
\NormalTok{input\_vector }\OtherTok{\textless{}{-}} \FunctionTok{c}\NormalTok{(}\StringTok{"Alex"}\NormalTok{, }\StringTok{"Bettina"}\NormalTok{, }\StringTok{"Christina"}\NormalTok{)  }\CommentTok{\# Define input vector}
\NormalTok{output\_permutation }\OtherTok{\textless{}{-}} \FunctionTok{draw\_names}\NormalTok{(input\_vector)  }\CommentTok{\# Draw names}
\FunctionTok{print}\NormalTok{(output\_permutation)  }\CommentTok{\# Print the result}
\end{Highlighting}
\end{Shaded}

\begin{verbatim}
## [1] "Bettina"   "Christina" "Alex"
\end{verbatim}

\subsubsection{Drawbacks of our
Solution}\label{drawbacks-of-our-solution}

The drawback is that for even 1 wrongly placed name we need to reshuffle
the whole permutation. If the list of names is considerably small then
this algorithm will finish in time. With larger datasets it is
considerably more expensive to reshuffle. Therefore one should use more
specialized algorithms.

One such algorithm would choose for each index a random entry from the
remainder of the names. If this one index matches the input one has to
just fetch another (e.g.~the next one). Thus, no complete reshuffling
takes place, making the algorithm quicker for larger datasets.

\end{document}
